\documentclass{article}
\usepackage{siunitx} 
\usepackage{graphicx}
\usepackage{natbib}
\usepackage{amsmath} 

\setlength\parindent{0pt}

\renewcommand{\labelenumi}{\alph{enumi}.}

%\usepackage{times} 

%----------------------------------------------------------------------------------------
%	DOCUMENT INFORMATION
%----------------------------------------------------------------------------------------

\title{Práctica 3 \\ Programación Concurrente y de Tiempo Real \\Universidad de Cádiz} % Title

\author{Alejandro Serrano Fernández} % Author name

\date{\today} % Date for the report

\begin{document}

\maketitle % Insert the title, author and date


%----------------------------------------------------------------------------------------
%	SECTION 1
%----------------------------------------------------------------------------------------

\section{Ejercicio 2}
Para este ejercicio escribiremos un programa secuencial y otro paralelo para efectuar el producto de dos matrices. En la siguiente tabla observamos los siguientes resultados obtenidos:

\hfill \break
\begin{center}
 \begin{tabular}{| c | c |}
 	Hilos & Tiempo Empleado (En milisegundos)\\
 	1 & 8.0 \\
 	1 & 10.0 \\
 	1 & 7.0 \\
 	2 & 16.0 \\
 	2 & 14.0 \\
 	2 & 12.0 \\
 	4 & 170 \\
 	4 & 19.0 \\
 	4 & 20.0 \\
 	6 & 18.0 \\
 	6 & 19.0 \\
 	6 & 20.0 \\
 	8 & 20.0 \\
 	8 & 22.0 \\
 	8 & 20.0 \\
 	10 & 18.0 \\
 	10 & 19.0 \\
 	10 & 20.0 \\
 \end{tabular}
\end{center}
\hfill \break
Como podemos observar, para mayores números de hilos, el tiempo de ejecución aumenta.
\end{document}




