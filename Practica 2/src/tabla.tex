\documentclass{article}
\usepackage{siunitx} 
\usepackage{graphicx}
\usepackage{natbib}
\usepackage{amsmath} 

\setlength\parindent{0pt}

\renewcommand{\labelenumi}{\alph{enumi}.}

%\usepackage{times} 

%----------------------------------------------------------------------------------------
%	DOCUMENT INFORMATION
%----------------------------------------------------------------------------------------

\title{Práctica 2 \\ Programación Concurrente y de Tiempo Real \\Universidad de Cádiz} % Title

\author{Alejandro Serrano Fernández} % Author name

\date{\today} % Date for the report

\begin{document}

\maketitle % Insert the title, author and date

% If you wish to include an abstract, uncomment the lines below
% \begin{abstract}
% Abstract text
% \end{abstract}

%----------------------------------------------------------------------------------------
%	SECTION 1
%----------------------------------------------------------------------------------------

\section{Ejercicio 1}
Para la realización del Ejercicio 1 he hecho uso de 4 hilos, que dependiendo de su identificador, serán pares o impares, luego tendremos 2 hilos con identificador par, y otros 2 con identificador impar. En el caso de que sea par, se incrementará en uno la variable n tantas veces como iteraciones haya. En caso contrario, se decrementará.
 \\
\begin{center}
 \begin{tabular}{|c| c|}
 	Iteraciones & Resultado \\
 	10000 & 1071	\\
 	10000 & 875	\\
 	20000 & -3048 	\\
 	20000 & 6284 	\\
 	1000 & 0	\\
 	1000 & 0 \\
 	1000 & 0 \\
 	500 & 0 \\
 	500 & 0 
 \end{tabular}
 
\end{center}
\hfill \break
Como podemos observar, para iteraciones de menor tamaño, devuelve el valor esperado, aunque para iteraciones de mayor tamaño podemos comprobar que los resultados son distintos de 0. 

%----------------------------------------------------------------------------------------
%	SECTION 2
%----------------------------------------------------------------------------------------

\section{Ejercicio 2}
Para este ejercicio emplearé la misma condición de concurso y el mismo número de hilos (2 para incrementar y 2 para decrementar).
\begin{center}
 \begin{tabular}{|c| c|}
 	Iteraciones & Resultado \\
 	10000 & 1104	\\
 	10000 & 1506	\\
 	20000 & -9330 	\\
 	20000 & 12619 	\\
 	1000 & 0	\\
 	1000 & 0 \\
 	1000 & 0 \\
 	500 & 0 \\
 	500 & 0 
 \end{tabular}
 
\end{center}
\hfill \break
Los resultados obtenidos son parecidos a la implementación del ejercicio 1. Para iteraciones menores obtenemos 0, que es el resultado teórico, y para iteraciones mayores, obtenemos resultados totalmente distintos al esperado.
\end{document}